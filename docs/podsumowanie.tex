\chapter{Podsumowanie}

Realizacja tego projektu nauczyła wiele na temat działania sieci neuronowych i zasad operowania na zbiorze uczącym. Okazało się, że algorytmy uczenia maszynowego mogą być zastosowane do stworzenia interaktywnej gry. Bardzo dużo zależy jednakże od dwóch głównych czynników: doboru odpowiednich parametrów sieci oraz posiadania odpowiedniego zbioru uczącego.

Parametry sieci można ustawić bazując na posiadanych doświadczeniach oraz przeprowadzając szereg testów. Możliwych parametrów jest bardzo wiele, lecz najważniejsze są dwa: liczba warstw ukrytych oraz liczba neuronów w każdej warstwie. Projekt pokazał, że nie ma potrzeby inwestowania w wiele warstw o ile zbiór jest odpowiednio dobrany, więc jedna warstwa jest wystarczająca dla tego problemu. Liczbę neuronów najlepiej jest wybierać sprawdzając efektywność algorytmu dla wielu różnych wartości z zadanego przedziału, najlepiej zaczynając od 1 neuronu. Efektywność może być miarą różnej postaci, w tym wypadku został zastosowany iloraz poprawnie odgadniętych państw przez całkowitą liczbę państw. Ważne jest tutaj także sprawdzenie ile czasu zajmuje uczenie algorytmu przy różnych liczbach neuronów, aby wybrać optymalną wartość.

Kolejną bardzo ważną kwestią jest dobranie odpowiedniego zbioru uczącego. W przypadku tego problemu operowano na zbiorze, w którym liczba próbek była taka sama jak liczba klas, co oznaczało że każda klasa charakteryzowała się tylko jednym opisem cech, algorytm zatem bardzo łatwo popadał w lokalne ekstrema. Aby poradzić sobie z tym problemem należy przede wszystkim usunąć wszystkie cechy nieskorelowane bezpośrednio z rozpatrywanym problemem - w tym wypadku dane nie określające flagi, lecz cechy kraju. Można także dodać do zbioru dodatkową liczbę wygenerowanych ręcznie próbek, które to bazują na istniejącym zbiorze.

Warto także mieć na uwadze, że wiele pakietów do uczenia maszynowego udostępnia mnogość różnych narzędzi do operowania czy testowania algorytmów. Bardzo przydatnymi może się okazać chociażby narzędzie do selekcji cech, które można użyć do sprawdzenia, które cechy liczą się najbardziej przy rozpatrywaniu przynależności do klasy. Pozwoli to na przykład usunąć najmniej potrzebne cechy ze zbioru. Przydatnym także jest rozkład pewności, za pomocą którego można sprawdzić z jaką pewnością algorytm wybrał przynależność do danej klasy.