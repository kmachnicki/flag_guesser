\chapter{Wstęp}

\section{Cele i założenia projektowe}

Celem projektu było zapoznanie się z tematem obliczeń neuronowych i zastosowanie ich do wybranego problemu. Realizacja projektu miała umożliwić zrozumienie zasady działania sieci neuronowych, wpływ zbioru uczącego oraz parametrów sieci na efektywność uczenia oraz sposoby testowania nauczonego algorytmu i generowanie zbioru testowego.

\section{Opis projektu}

Jako przedmiot zainteresowania wybrane zostało zagadnienie stworzenia gry typu człowiek-komputer, w której to gracz wybierałby flagę istniejącego obecnie państwa, zaś algorytm zadając odpowiednie pytania starałby się zgadnąć flaga którego państwa została wybrana.

Głównym założeniem projektu stało się sprawdzenie, czy sieć neuronowa sprawdzi się przy zagadnieniu stworzenia w miarę interaktywnej rozgrywki z graczem oraz czy nauczony algorytm wykazałby się wysoką dokładnością przy zgadywaniu. Problemem tutaj byłoby odpowiednie dobranie parametrów sieci tak, aby jej efektywność w zgadywaniu była największa.

Kolejnym wyzwaniem było odpowiednie dobranie zbioru uczącego. Zbiór taki musiał zapewnić maksymalną efektywność w odnajdywaniu odpowiedzi oraz posiadać jak najmniejszą liczbę nieskorelowanych danych, nie związanych z rozpatrywanym problemem.

Jako dodatkowy cel postawiono również na stworzenie graficznego interfejsu dla aplikacji, aby ułatwić interakcję gracza z komputerem.